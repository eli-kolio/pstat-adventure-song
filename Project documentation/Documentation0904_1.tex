\documentclass[12pt, twoside, a4paper ,openright]{book}
\usepackage{calc}
\usepackage{amsfonts}
\usepackage{latexsym}
\usepackage{placeins}
\ifx\pdftexversion\undefined
  \usepackage[dvips]{graphicx}
\else
  \usepackage[pdftex]{graphicx}
\fi
\usepackage{amsthm}
\usepackage{amssymb}
\usepackage{mathrsfs}
\usepackage{authblk}
\usepackage{amsmath}
\usepackage[T2A]{fontenc}
\usepackage[utf8]{inputenc}
\usepackage[bulgarian]{babel}
\usepackage{graphicx}
\usepackage{amsfonts}
\usepackage{pdfpages}
\usepackage{dsfont}
\usepackage{enumerate}
\addtolength{\topmargin}{-1cm}
\let\tmp\oddsidemargin
\let\oddsidemargin\evensidemargin
\let\evensidemargin\tmp
\reversemarginpar
\usepackage{fancyhdr}
\pagestyle{fancy}
\renewcommand{\chaptermark}[1]{\markboth{#1}{}}
\renewcommand{\sectionmark}[1]{\markright{\thesection\ #1}}
\fancyhf{}
\fancyhead[LE,RO]{\bfseries\thepage}
\fancyhead[LO]{\bfseries\rightmark}
\fancyhead[RE]{\bfseries\leftmark}
\renewcommand{\headrulewidth}{0.5pt}
\renewcommand{\footrulewidth}{0pt}
\addtolength{\headheight}{0.5pt}
\fancypagestyle{plain}{
\fancyhead{}
\renewcommand{\headrulewidth}{0pt}
}
\newtheorem{liuvil}{Теорема}
\newtheorem*{lax}{Твърдение}
\newtheorem*{laxhamilton}{Твърдение}
\newtheorem*{todalax}{Твърдение}
\newtheorem*{rtoda}{Твърдение}
\newtheorem*{jacobi}{Твърдение}
\newtheorem*{toda}{Твърдение}
\newtheorem*{todasolution}{Твърдение}
\newtheorem*{tvardenie1}{Твърдение}
\newtheorem*{tvardenie2}{Твърдение}
\newtheorem*{lema}{Лема}
\newtheorem*{lemaa}{Лема}
\theoremstyle{definition}
\newtheorem*{grupa}{Дефиниция}
\newtheorem*{algebra}{Дефиниция}



\begin{document}

\title{\LARGE Проект по Приложна Статистика \\ [1em] \huge \textbf{Анализ на музикални данни от "Million Songs"}}
\author{Никола Юруков \hspace{1cm} Елица Илиева}
\date{\\[1em] {9 април 2017 г.} \\ [0.8cm] Преподавател: г. ас. д-р Нина Даскалова\\ [0.5cm] Софийски Университет "Св. Климент Охридски"  \\ Факултет по Математика и Информатика
	
\clearpage
\maketitle
\tableofcontents
\thispagestyle{empty}
\renewcommand\abstractname{Abstract}
\renewcommand\tablename{Table}
\newpage
\addcontentsline 

\chapter{Въведение}
	Работим върху датасет, който съдържа информация за повече от милион песни, кога са издадени, кой е техният автор, каква тоналност имат, дали може да се танцува на тях, и какъв е техния "hotness". На базата на статистичеки проучвания целим да намерим корелацията между това колко харесвана е дадена песен спрямо нейната ритмичност и редица други подобни показатели. 
	
\section{Защо избрахме този дейтасет?}



\section{Какви данни съдържа той?}

\chapter{Заключение}

\begin{thebibliography}{99}
\small
\bibitem[1]{arnold} V. Arnold, Mathematical Methods of Classical Mechanics 2nd ed. (1988)
\bibitem[2]{babelon} O. Babelon, D. Bernard, M. Talon,  Introduction to Classical Integrable Systems (2003)
\bibitem[3]{nakahara} M. Nakahara, Geomerty, Topology and Physics (1989)
\bibitem[4]{das} A. Das, Integrable Models (1989)
\bibitem[5]{gilmore} R. Gilmore, Lie groups, Physics, and Geometry: An Introduction for Physicists, Engineers and Chemists (2008)
\bibitem[6]{szekeres} P. Szekeres, A Course in Modern Mathematical Physics -- Groups, Hilbert Spaces and Diff. Geometry (2004)
\bibitem[7]{isham} C. Isham, Modern Differential Geometry for Physicists (1999)
\bibitem[8]{toda} M. Toda, Theory of nonlinear lattices (1981)
\bibitem[9]{knapp} A. W. Knapp, Structure Theory of Semisimple Lie Groups (1997)
\bibitem[10]{perelomov1} M. A. Olshanetsky, A. M. Perelomov, Classical Integrable Finite-Dimensional Systems Related to Lie Algebras (1980)
\bibitem[11]{perelomov2} M. A. Olshanetsky, A. M. Perelomov, Explicit Solutions of Classical Toda Models (1979)
\bibitem[12]{kodama} Y. Kodama, J. Ye, The Generalized Toda Lattice Equation on Semisimple Lie Algebras (1995)
\bibitem[13]{krichever} I. Krichever, K. Vaninsky, The Periodic and Open Toda Lattice (2000)
\end{thebibliography}
\end{document} 